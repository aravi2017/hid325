\documentclass[sigconf]{acmart}

\usepackage{hyperref}

\usepackage{endfloat}
\renewcommand{\efloatseparator}{\mbox{}} % no new page between figures

\usepackage{booktabs} % For formal tables

\settopmatter{printacmref=false} % Removes citation information below abstract
\renewcommand\footnotetextcopyrightpermission[1]{} % removes footnote with conference information in first column
\pagestyle{plain} % removes running headers

\begin{document}
\title{Impact of Big Data on the Privacy of Mental Health Patients}


\author{J. Robert Langlois}
\affiliation{%
 \institution{Indiana University Bloomington, School of Informatics and Computing}
 }
\email{langloir@umail.iu.edu}

\begin{abstract}
Thanks to the rise of technology, the health-care field has experienced an increase of health information on a daily basis. This increases in health information technology and electronic records have the potential to improve clinical research. However, privacy remain a serious impediment to big data that needs to be addressed. By resolving patients privacy concerns, policymakers and researchers can help transform the mental health field, avoid unnecessary expenses, and establish proper norms to communicate sensitive health information.    
\end{abstract}



\maketitle

\section{Introduction}

We live in an era of big data; data exists everywhere in large quantity. The advances in technology has opened the door for businesses to collect inconceivable amount of pieces of information on individuals via emails, smart-phones, sensors, and other social media. The 21st century has witnessed a data explosion; many fields have experienced a data deluge that can contribute to boast the economy via data analysis, make new discovery based on existing data, respond to health problems in quickly manner, and so forth. While it worth celebrating the rapid innovation of technology and the presence of data deluge, it is also crucial to consider the number of barriers and risks that come with the increase of big data. One of the barriers that big data faces is privacy. In the health-care industry, for example, it is not easy to access data due to privacy concern; thus policymakers need to establish proper norms and parameters to collect, share, and house the data. ''When considering the risks that big data poses to individual privacy, policymakers should be mindful of its sizable benefits''\cite{tene2012big}. While it is important to address the numerous advantages of big data, it remains relevant to figure out ways to prevent data leakage, and to protect the privacy of individuals. This paper showcases the advantages of big data and the ways to overcome the individual privacy concerns.    

\section{Advantages of Big Data}

Big data presents a number of advantages. Big data helps businesses increase their productivity, it allows government to improve public sector administration and assists global organization in analyzing information. Big data can help to detect disease at an early stage and reduce the effect of seasonal disease on individual. Other advantages of big data analysis is present in many different areas, such as:  smart grid, which helps to monitor and control electricity use; traffic management, which provides information on road and mass transit construction, traffic congestion; retail by studying customers behavior to improve store layout; payments by helping to detect fraud detection, etc.\cite{tene2012big}.

Certain researchers supported the idea that big data allow real time tracking of diseases, predicting outbreaks, and developing personalized healthcare. Big data can really help to maximize profits in many disciplines, including healthcare if harness properly.\cite{van2011health}. As indicates in \cite{khan2014big} ''by harnessing big data, businesses gain many advantages, including increased operational efficiency, informed strategic direction, improved customer service, new products, and new customers and markets.'' While data exists in huge quantity in many fields, including the health care field. Individual Privacy remains a big problem that policymakers have to tackle to have proper access in the health care industry.

\section{Barriers to Big Data in Health-Care}

One of the barriers to big data in the health care, including mental health is privacy. Regardless of the effort of Policymakers to try to establish different strategies to protect individual health information, privacy remains a serious issue that scientists have to wrestle with when it comes to big data analytic. among the effort of policy makers to secure health information, they have created, for example, Health Insurance Portability and Accountability Act of 1996 HIPAA established the norms to data privacy and security provisions for safeguarding medical and mental health information. Every provider in the healthcare industry must obey the HIPAA privacy laws if they want to continue to remain up and running. The HIPAA laws prohibit providers to share patient's information without their consent, and a lot of time patients refuse to share their personal information for research purposes by fear of being ostracized, discriminated against, marginalized, etc. ''The unintended release of a person’s health information into the public realm has huge potential to undermine personal dignity and cause embarrassment and financial harm''\cite{van2011health}. While the healthcare field is faced with a huge increase in health information, individual privacy concern remain a huge conundrum to big data analysis. What can policymakers do to overcome this privacy concern?

\section{Ways to Overcome privacy concern}
\subsubsection{Data Anonymization}
One way policymakers can protect individual privacy is by making the data anonymous. Researchers identified three types of data: personal and proprietary data that is controlled by individuals, government controlled data, which government can restrict access, and open data commons, which means that the data is available to all. They advocated for linking data together that can help to improve care planning at both the patient and population levels. They also argued for an increase of the amount of information that is available as open data commons. Though anonymization of data appears to be a great technique that policy makers could espouse to address the privacy concern, other study indicated that the data can be replaced back to their respective individual.\cite{van2011health}. '' Every copy of data increases the risk of unintended disclosure. To reduce this risk, data should be anonymized before transfer; upon receipt, the recipient will have no choice but anonymize it at rest...And re-identification is by design, in order to ensure accountability, reconciliation and audit.'' If proper norms is established for data analysis, this will contribute to improve the health care industry. 

Others advocated for data de-identification and data minimization. The term de-identification is the process of making the data anonymous, but these author explained that this protective measure is valid under the security and accountability principles, but policymakers should think about other ways to protect patient’s privacy. The term data minimization, is the extent to which organizations limit the collection of personal data. It worth noting that data minimization is contrary to big data analysis because data minimization encourages deleting data that is no longer in use to protect privacy, whereas big data prefers to archive the data for ulterior usage. While this technique can help protect privacy, it is antithetic to big data analysis because it contributes to reduce the amount of data collection that could be utilized in data analysis to make new discovery, respond to crisis, and maximize profits \cite{tene2012big}. 
As found in \cite{cavoukian2012privacy} privacy principles should be introduced during data architecture; privacy should be incorporate in the design and the operation procedures. in so doing, personal health care data will be protected against malicious hackers who always try to access individuals personal health information for the purposes of stealing individual's identity. 

The concept quantified self can be understood by the fact that individuals engage in self-tracking of personal health data, such as heart rate, weight, energy level, sleep quality, cognitive performance, etc. these individuals use devices like smart-phones, watches, sensors, in the collection of their personal data. It has shown that 60 percent of us adult are tracking their weight, diet or exercise routine, 33 percent are monitoring their blood sugar, blood pressure, sleep patterns, etc. this indicates that there is a vast amount of health that has been collected by individuals; this demonstrates the need to develop policies and involve individual patients into sharing their data to advances health-care through data analysis. Before we talk about analyzing data, the norm to use these data need to be established primarily. \cite{swan2013quantified}.

as found in 



\section{Conclusion}

We have seen that health data exist in large quantity; however, privacy concern is one of the biggest barriers that scientists face when it comes to utilize of health data. Certain researchers proposed data anonymization as a solution to privacy concern, others proposed minimization of the amount of data collected on individual patients. ''Privacy concerns exist wherever personally identifiable information or other sensitive information is collected and stored in any form.''\cite{khan2014big} This indicates that scientists will allow wrestle with privacy concern whenever they are dealing with personal health information.  



\appendix

%Appendix A
\section{Headings in Appendices}

the body of this document in Appendix-appropriate form:

\subsection{Introduction}
\subsection{The Body of the Paper}
\subsubsection{Type Changes and  Special Characters}
\subsubsection{Math Equations}
\paragraph{Inline (In-text) Equations}
\paragraph{Display Equations}
\subsubsection{Citations}
\subsubsection{Tables}
\subsubsection{Figures}
\subsubsection{Theorem-like Constructs}
\subsubsection*{A Caveat for the \TeX\ Expert}
\subsection{Conclusions}
\subsection{References}

\texttt{.bbl} file.  Insert that \texttt{.bbl} file into the
\texttt{.tex} source file and comment out the command
\texttt{{\char'134}thebibliography}.

% This next section command marks the start of
% Appendix B, and does not continue the present hierarchy


\bibliographystyle{ACM-Reference-Format}
\bibliography{report} 



\end{document}
